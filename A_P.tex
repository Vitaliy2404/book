\documentclass{article}
\usepackage{graphicx}
\usepackage{subcaption}
\usepackage[T2A]{fontenc}
\usepackage[utf8x]{inputenc}
\usepackage[english, ukrainian]{babel}
\usepackage{url}
\usepackage{latexsym,amsmath,amsfonts,amssymb,amsthm,mathrsfs}
\usepackage{cite,hhline}
\usepackage{hyperref}
\usepackage{graphicx,mwe}
\usepackage{wrapfig}
\usepackage{array}
\usepackage{tabularx, multirow}
\usepackage{longtable}
\usepackage{lipsum} % just for dummy text- not needed for a longtable
%\usepackage{minted}

\addto\captionsukrainian{\renewcommand{\figurename}
\usepackage{book}					% стильовий файл журналу 
\title{План-конспект уроку інформатики для учнів 2 класу}
\date{2022-12-10}
\author{Повар Аліна Вікторівна}
\begin{document}
	\maketitle
	\newpage
	\Title{План-конспект уроку інформатики}\label{radap1xxx:FirstPage}
\author{Підготувала: Повар Аліна Вікторівна}
\aff{вчитель Ліцею 2 імені Л.Х.Дарбіняна Коростишівської міської ради, Житомирська обл. м.Коростишів, Україна}
\Address{{\href{alina.povar@gmail.com}}
\section{Хід уроку}
\begin{center} І.. ЕТАП ОРІЄНТАЦІЇ, МОТИВАЦІЇ ДІЯЛЬНОСТІ \end{center}
Урок розпочинається під спокійну музику 
\begin {enumerate}
\item привітання настановка на урок (віршоване)
	\item перевірка готовності учнів до уроку
	\item перевірка присутніх;
	\item Мотивація 
\end {enumerate}
\begin {center} І. ЕТАП ЦІЛЕВИЗНАЧЕННЯ І ПЛАНУВАННЯ   \end{center}
\begin {enumerate}
	\item Повідомлення теми, мети уроку
	\item Актуалізація знань, особистого досвіду учнів. Завдання з країни «Найрозумніший» Ребус-ключове слово уроку (на слайді/в друк.зош)
	\item Формулювання очікуваних результатів.
\end {enumerate}
\begin {center} ІІІ. ЕТАП ЦІЛЕРЕАЛІЗАЦІЇ    \end{center}
\begin {enumerate}
	\item Пояснення вчителя з елементами демонстрування презентації
	\item Фізкультхвилинка
	\item Робота з зошитом (відкриття 14)
	\item Робота за комп’ютером 
Повторення правил безпечної поведінки за комп’ютером.
Інструктаж учителя. (Робота з підручником: § 14 ст. 90-91)
Практична робота за комп’ютерами в середовищі програми MO Excel «Таблиця множення»
	\item ДОДАТКОВО. Практична робота за комп’ютерами в середовищі програми MO Excel «Площа огорожі»; «Продукти»
	\item Інтерактивна вправа «Введення даних в електронну таблицю»  https://learningapps.org/20743629 
	\item Вправи для очей
\end {enumerate}
\begin {center} ІV. ЕТАП РЕФЛЕКСИВНО-ОЦІНЮВАЛЬНИЙ. \end{center}
\begin {enumerate}
	\item {Підбиття підсумків уроку.}
	\item Оцінюємо свої знання і вміння
Я використовую комп’ютерні програми для створення математичних моделей, зокрема програму яку використовують для опрацювання даних в таблицях.
Я прогнозую і формулюю очікуваний результат виконання формули.
	\item Оцінювання навчальних досягнень учнів.
	\item Рефлексія. проходить під спокійну мелодію 
Подумайте як ви працювали на уроці, дайте відповіді на запитання та проаналізуйте чи все у вас вдалося.
\end{enumerate}
\end{document}